\documentclass[12pt,a4paper]{article}
\usepackage{amsmath}
\usepackage{amssymb}
\usepackage{geometry}
\usepackage{graphicx}
\usepackage{svg}
\geometry{margin=1in}

\title{JEE Mathematics Mock Test Questions}
\author{Osmium AI}
\date{}

\begin{document}

\maketitle

\section*{Mathematics Questions (25)}

\subsection*{Question 1}
Evaluate the following integral:
\[
\int_{0}^{\pi} \sin(x) \, dx
\]
\textbf{Options:}
\begin{itemize}
    \item[(A)] 0
    \item[(B)] 1
    \item[(C)] 2
    \item[(D)] $\pi$
\end{itemize}
\textbf{Answer: C}

\subsection*{Question 2}
If $f(x) = x^3 - 3x^2 + 4$, then $f'(2)$ is:
\[
f'(x) = 3x^2 - 6x
\]
\textbf{Options:}
\begin{itemize}
    \item[(A)] 0
    \item[(B)] 4
    \item[(C)] 6
    \item[(D)] 12
\end{itemize}
\textbf{Answer: A}

\subsection*{Question 3}
The value of the limit:
\[
\lim_{x \to 0} \frac{\sin x}{x}
\]
\textbf{Options:}
\begin{itemize}
    \item[(A)] 0
    \item[(B)] 1
    \item[(C)] $\infty$
    \item[(D)] Does not exist
\end{itemize}
\textbf{Answer: B}

\subsection*{Question 4}
If $A = \begin{bmatrix} 1 & 2 \\ 3 & 4 \end{bmatrix}$, then $\det(A)$ is:
\[
\det(A) = (1)(4) - (2)(3)
\]
\textbf{Options:}
\begin{itemize}
    \item[(A)] $-2$
    \item[(B)] 2
    \item[(C)] 10
    \item[(D)] $-10$
\end{itemize}
\textbf{Answer: A}

\subsection*{Question 5}
The number of ways to arrange 5 different books on a shelf is:
\[
5! = 5 \times 4 \times 3 \times 2 \times 1
\]
\textbf{Options:}
\begin{itemize}
    \item[(A)] 25
    \item[(B)] 60
    \item[(C)] 120
    \item[(D)] 720
\end{itemize}
\textbf{Answer: C}

\subsection*{Question 6}
If $P(A) = 0.6$ and $P(B) = 0.4$, and $A$ and $B$ are independent, then $P(A \cap B)$ is:
\[
P(A \cap B) = P(A) \times P(B)
\]
\textbf{Options:}
\begin{itemize}
    \item[(A)] 0.24
    \item[(B)] 0.4
    \item[(C)] 0.6
    \item[(D)] 1.0
\end{itemize}
\textbf{Answer: A}

\subsection*{Question 7}
The equation of a circle with center $(3, 4)$ and radius 5 is:
\[
(x - 3)^2 + (y - 4)^2 = r^2
\]
\textbf{Options:}
\begin{itemize}
    \item[(A)] $(x-3)^2 + (y-4)^2 = 5$
    \item[(B)] $(x-3)^2 + (y-4)^2 = 25$
    \item[(C)] $(x+3)^2 + (y+4)^2 = 25$
    \item[(D)] $x^2 + y^2 = 25$
\end{itemize}
\textbf{Answer: B}

\subsection*{Question 8}
The sum of first $n$ natural numbers is:
\[
S_n = \frac{n(n+1)}{2}
\]
\textbf{Options:}
\begin{itemize}
    \item[(A)] $n(n+1)$
    \item[(B)] $\frac{n(n+1)}{2}$
    \item[(C)] $n^2$
    \item[(D)] $2n$
\end{itemize}
\textbf{Answer: B}

\subsection*{Question 9}
If $\log_{10} 2 = 0.3010$, then $\log_{10} 8$ is:
\[
\log_{10} 8 = \log_{10} 2^3 = 3 \log_{10} 2
\]
\textbf{Options:}
\begin{itemize}
    \item[(A)] 0.6020
    \item[(B)] 0.9030
    \item[(C)] 2.4080
    \item[(D)] 0.3010
\end{itemize}
\textbf{Answer: B}

\subsection*{Question 10}
The roots of the quadratic equation:
\[
x^2 - 5x + 6 = 0
\]
\textbf{Options:}
\begin{itemize}
    \item[(A)] 2, 3
    \item[(B)] $-2$, $-3$
    \item[(C)] 1, 6
    \item[(D)] $-1$, $-6$
\end{itemize}
\textbf{Answer: A}

\subsection*{Question 11}
The value of the definite integral:
\[
\int_{0}^{1} x^2 \, dx = \left[ \frac{x^3}{3} \right]_{0}^{1}
\]
\textbf{Options:}
\begin{itemize}
    \item[(A)] $\frac{1}{2}$
    \item[(B)] $\frac{1}{3}$
    \item[(C)] $\frac{1}{4}$
    \item[(D)] 1
\end{itemize}
\textbf{Answer: B}

\subsection*{Question 12}
If $\sin \theta = \frac{3}{5}$, then $\cos \theta$ is ($\theta$ in first quadrant):
\[
\cos^2 \theta = 1 - \sin^2 \theta = 1 - \frac{9}{25} = \frac{16}{25}
\]
\textbf{Options:}
\begin{itemize}
    \item[(A)] $\frac{3}{5}$
    \item[(B)] $\frac{4}{5}$
    \item[(C)] $\frac{5}{3}$
    \item[(D)] $\frac{5}{4}$
\end{itemize}
\textbf{Answer: B}

\subsection*{Question 13}
The distance between points $(1, 2)$ and $(4, 6)$ is:
\[
d = \sqrt{(x_2 - x_1)^2 + (y_2 - y_1)^2} = \sqrt{9 + 16}
\]
\textbf{Options:}
\begin{itemize}
    \item[(A)] 3
    \item[(B)] 4
    \item[(C)] 5
    \item[(D)] 7
\end{itemize}
\textbf{Answer: C}

\subsection*{Question 14}
The slope of the line $3x + 4y = 12$ is:
\[
4y = -3x + 12 \implies y = -\frac{3}{4}x + 3
\]
\textbf{Options:}
\begin{itemize}
    \item[(A)] $\frac{3}{4}$
    \item[(B)] $-\frac{3}{4}$
    \item[(C)] $\frac{4}{3}$
    \item[(D)] $-\frac{4}{3}$
\end{itemize}
\textbf{Answer: B}

\subsection*{Question 15}
If vectors $\vec{a}$ and $\vec{b}$ are perpendicular, then $\vec{a} \cdot \vec{b}$ is:
\[
\vec{a} \cdot \vec{b} = |\vec{a}||\vec{b}|\cos 90° = 0
\]
\textbf{Options:}
\begin{itemize}
    \item[(A)] 1
    \item[(B)] $-1$
    \item[(C)] 0
    \item[(D)] $|\vec{a}||\vec{b}|$
\end{itemize}
\textbf{Answer: C}

\subsection*{Question 16}
The area of a triangle with vertices $(0,0)$, $(3,0)$, $(0,4)$ is:
\[
\text{Area} = \frac{1}{2} \times \text{base} \times \text{height} = \frac{1}{2} \times 3 \times 4
\]
\textbf{Options:}
\begin{itemize}
    \item[(A)] 6
    \item[(B)] 12
    \item[(C)] 7
    \item[(D)] 24
\end{itemize}
\textbf{Answer: A}

\subsection*{Question 17}
The value of $i^2$ (where $i = \sqrt{-1}$) is:
\[
i^2 = (\sqrt{-1})^2 = -1
\]
\textbf{Options:}
\begin{itemize}
    \item[(A)] 1
    \item[(B)] $-1$
    \item[(C)] $i$
    \item[(D)] 0
\end{itemize}
\textbf{Answer: B}

\subsection*{Question 18}
The number of diagonals in a hexagon is:
\[
\text{Diagonals} = \frac{n(n-3)}{2} = \frac{6(6-3)}{2} = \frac{18}{2}
\]
\textbf{Options:}
\begin{itemize}
    \item[(A)] 6
    \item[(B)] 9
    \item[(C)] 12
    \item[(D)] 15
\end{itemize}
\textbf{Answer: B}

\subsection*{Question 19}
If $f(x) = e^x$, then $f'(x)$ is:
\[
\frac{d}{dx}(e^x) = e^x
\]
\textbf{Options:}
\begin{itemize}
    \item[(A)] $e^x$
    \item[(B)] $xe^{x-1}$
    \item[(C)] $e^{x-1}$
    \item[(D)] 1
\end{itemize}
\textbf{Answer: A}

\subsection*{Question 20}
The mean of the data set $\{2, 4, 6, 8, 10\}$ is:
\[
\text{Mean} = \frac{2 + 4 + 6 + 8 + 10}{5} = \frac{30}{5}
\]
\textbf{Options:}
\begin{itemize}
    \item[(A)] 5
    \item[(B)] 6
    \item[(C)] 7
    \item[(D)] 8
\end{itemize}
\textbf{Answer: B}

\subsection*{Question 21}
The value of $\tan 45°$ is:
\[
\tan 45° = \frac{\sin 45°}{\cos 45°} = \frac{\frac{1}{\sqrt{2}}}{\frac{1}{\sqrt{2}}}
\]
\textbf{Options:}
\begin{itemize}
    \item[(A)] 0
    \item[(B)] 1
    \item[(C)] $\sqrt{3}$
    \item[(D)] $\frac{1}{\sqrt{3}}$
\end{itemize}
\textbf{Answer: B}

\subsection*{Question 22}
If $A$ and $B$ are mutually exclusive events, then $P(A \cup B)$ is:
\[
P(A \cup B) = P(A) + P(B) - P(A \cap B) = P(A) + P(B)
\]
\textbf{Options:}
\begin{itemize}
    \item[(A)] $P(A) \times P(B)$
    \item[(B)] $P(A) + P(B)$
    \item[(C)] $P(A) - P(B)$
    \item[(D)] 0
\end{itemize}
\textbf{Answer: B}

\subsection*{Question 23}
The derivative of $\ln x$ is:
\[
\frac{d}{dx}(\ln x) = \frac{1}{x}
\]
\textbf{Options:}
\begin{itemize}
    \item[(A)] $\frac{1}{x}$
    \item[(B)] $x$
    \item[(C)] $\ln x$
    \item[(D)] $e^x$
\end{itemize}
\textbf{Answer: A}

\subsection*{Question 24}
The sum of angles in a triangle is:
\[
\angle A + \angle B + \angle C = 180°
\]
\textbf{Options:}
\begin{itemize}
    \item[(A)] $90°$
    \item[(B)] $180°$
    \item[(C)] $270°$
    \item[(D)] $360°$
\end{itemize}
\textbf{Answer: B}

\subsection*{Question 25}
If $|x| = 5$, then $x$ can be:
\[
|x| = 5 \implies x = \pm 5
\]
\textbf{Options:}
\begin{itemize}
    \item[(A)] Only 5
    \item[(B)] Only $-5$
    \item[(C)] 5 or $-5$
    \item[(D)] 0
\end{itemize}
\textbf{Answer: C}

\vspace{1cm}

\section*{Example Images from Wolfram MathWorld}

\subsection*{Ellipse Bipolar}
\begin{figure}[h]
\centering
\includegraphics[width=0.6\textwidth]{https://mathworld.wolfram.com/images/eps-svg/EllipseBipolar_700.svg}
\caption{Ellipse Bipolar Diagram}
\end{figure}

\subsection*{Catastrophe Inline Equation}
\begin{figure}[h]
\centering
\includegraphics[width=0.4\textwidth]{https://mathworld.wolfram.com/images/equations/Catastrophe/Inline5.svg}
\caption{Catastrophe Equation}
\end{figure}

\section*{Example: Archimedes' Axiom}

Symbolically, the axiom states that:
\[
\frac{a}{b} = \frac{c}{d}
\]
if and only if the appropriate one of following conditions is satisfied for integers $m$ and $n$:

\begin{enumerate}
    \item If $ma < nb$, then $mc < nd$.
    \item If $ma = nb$, then $mc = nd$.
    \item If $ma > nb$, then $mc > nd$.
\end{enumerate}

Formally, Archimedes' axiom states that if $AB$ and $CD$ are two line segments, then there exist a finite number of points $A_1, A_2, \ldots, A_n$ on $\overline{AB}$ such that:
\[
CD = AA_1 = A_1A_2 = \cdots = A_{n-1}A_n
\]

\end{document}
